\documentclass[a4paper]{scrreprt}

% Uncomment to optimize for double-sided printing.
% \KOMAoptions{twoside}

% Set binding correction manually, if known.
% \KOMAoptions{BCOR=2cm}

% Localization options
\usepackage[english]{babel}
\usepackage[T1]{fontenc}
\usepackage[utf8]{inputenc}

% Quotations
\usepackage{dirtytalk}

% Enhanced verbatim sections. We're mainly interested in
% \verbatiminput though.
\usepackage{verbatim}

% Automatically remove leading whitespace in lstlisting
\usepackage{lstautogobble}

% PDF-compatible landscape mode.
% Makes PDF viewers show the page rotated by 90°.
\usepackage{pdflscape}

% Advanced tables
\usepackage{array}
\usepackage{tabularx}
\usepackage{longtable}

% Fancy tablerules
\usepackage{booktabs}

% Graphics
\usepackage{graphicx}

% Current time
\usepackage[useregional=numeric]{datetime2}

% Float barriers.
% Automatically add a FloatBarrier to each \section
\usepackage[section]{placeins}

% Custom header and footer
\usepackage{fancyhdr}

\usepackage{geometry}
\usepackage{layout}

% Math tools
\usepackage{mathtools}
% Math symbols
\usepackage{amsmath,amsfonts,amssymb}
\usepackage{amsthm}

\DeclarePairedDelimiter{\ceil}{\lceil}{\rceil}
\DeclarePairedDelimiter{\floor}{\lfloor}{\rfloor}

% General symbols
\usepackage{stmaryrd}

\DeclarePairedDelimiter\abs{\lvert}{\rvert}

% Indistinguishable operator (three stacked tildes)
\newcommand*{\diffeo}{% 
  \mathrel{\vcenter{\offinterlineskip
  \hbox{$\sim$}\vskip-.35ex\hbox{$\sim$}\vskip-.35ex\hbox{$\sim$}}}}

% Bullet point
\newcommand{\tabitem}{~~\llap{\textbullet}~~}

\pagestyle{plain}
% \fancyhf{}
% \lhead{}
% \lfoot{}
% \rfoot{}
% 
% Source code & highlighting
\usepackage{listings}

% Coloured boxes!
\usepackage[most]{tcolorbox}
\newtcolorbox{library}[2][]{
	enhanced,
	sharp corners,
	coltitle=black, % title colour
	colbacktitle=black!10!white, % title bg colour
	halign title=center, % title align
	toptitle=1mm, % Top/bottom additional spacing for title
	bottomtitle=1mm,
	fonttitle=\ttfamily,
	colback=white, % body bg colour
	fontupper=\ttfamily,
	title=#2,#1
}

\newtcolorbox{boxcomment}[2][]{
	enhanced,
	colframe=white, % frame colour
	colbacktitle=white, % title bg colour
	halign=center, % body align
	colback=white, % body bg colour
	fonttitle=\ttfamily,
	fontupper=\ttfamily,
	title=#2,#1
}

% SI units
\usepackage[binary-units=true]{siunitx}
\DeclareSIUnit\cycles{cycles}

% Convenience commands
\newcommand{\mailsubject}{41100 - Distributed algorithms - Series 4}
\newcommand{\maillink}[1]{\href{mailto:#1?subject=\mailsubject}
                               {#1}}

% Should use this command wherever the print date is mentioned.
\newcommand{\printdate}{\today}

\subject{41102 - Distributed algorithms}
\title{Series 4}

\author{Michael Senn \maillink{michael.senn@students.unibe.ch} - 16-126-880}

\date{\printdate}

% Needs to be the last command in the preamble, for one reason or
% another. 
\usepackage{hyperref}


\begin{document}
\maketitle


\setcounter{chapter}{3}

\chapter{Series 4}

\section{Emulating a $(1, N)$ register from $(1, 1)$ registers}

\subsection{Emulating a safe binary $(1, N)$ register from safe binary $(1, 1)$ registers}

In order to be a safe binary register, a read on $orn$ must, if there is no
concurrent write, return the most recently written value. 

This clearly holds as $orn$ does not trigger the $WriteReturn$ event -
indicating that the write finished - before the last $(1, 1)$ register
indicates that the write finished. As such, any subsequent non-concurrent read
will return the most recently written value.

As such, $onr$ is a safe $(1, N)$ register.

\subsection{Emulating a regular binary $(1, N)$ register from regular binary $(1, 1)$ registers}
\label{seq:regular_binary_register}

In order to be a regular register, a read on $onr$ must, if there is no
concurrent write, return the most recently written value. If there is a
concurrent write it must return either the most recently written value, or one
of the values being currently written.

The first case clearly holds, with identical reasoning as above. The later
holds as, if any process $q$ reads from $onr$ during a concurrent write, it
indirectly reads from $br.q$. If $br.q$ already finished its write operation it
will return the value of the - from the point of view of $onr$ - concurrent
write. if $br.q$ did not finish its write operation yet then, due to being a
regular array, it is guaranteed to return either the value it stored previously
or the value which is being written to $br.q$. These are, from the POV of
$onr$, either the value of the most recently cmopleted write, or the value of
the concurrent write.

As such, $onr$ is a regular $(1, N)$ register.

\subsection{Emulating a regular multi-valued $(1, N)$ register from regular multi-valued $(1, 1)$ registers}

As the $(1, N)$ register $orn$ keeps one copy of its stored data in each of the
$(1, 1)$ registers $br.q$, rather than the data being spread across multiple
registers, orn is a regular multi-valued $(1, N)$ register using the exact same
reasoning as in section \ref{seq:regular_binary_register}.

If there is no concurrent write, the value of the most recently completed write
operation is returned. If there is a concurrent write then, due to $br.q$ being
a regular register, the returned value is either the one of the most recently
completed, or of the concurrent write operation.

As such, $onr$ is a regular multi-valued $(1, N)$ register.


\section{Multivalued register from binary registers}

Let $br.i, i \in \{1, \ldots k\}$ be regular binary $(1, 1)$ registers. We
implement a regular multi-valued $(1, 1)$ register as follows:

\begin{library}{Implementing regular multi-valued register $rr$}
  \begin{lstlisting}[mathescape=true,autogobble=true,breaklines=true]
    init:
      for i := 1 to k:
        br.i = 0

    func rr.read():
      for i := 1 to k:
        if br.i == 1:
	  return i
      return 0

    func rr.write(v):
      br.v = 1
      for i := v - 1 to 1:
        br.i = 0
  \end{lstlisting}
\end{library}

Specifically we represent a value $v$ such that the v-th register contains $1$,
while all other registers at lower indices contain $0$. Writing happens
right-to-left, reading happens left-to-right.

If there is no concurrent write then a read will, clearly, return the value of
the most recently completed write operation.

Consider the case of a concurrent write operation. Let $v$ be the value which
was written to $rr$ previously, and $v'$ be the value which is being written by
the write operation which is concurrent to the read operation. Let $i$ be the
index of the register which the read operation is currently reading.

It immediately follows that $i \leq \max\{v\}$, as the read operation
returns on the first $1$ it encounters. This leaves three situations to
consider: $i \leq v \leq v'$, $i \leq v' \leq v$, and $v' \leq i \leq v$.

\subsection{$i \leq v \leq v'$}

If the write process sets $br.v = 0$ before the read process reaches $br.v$,
then the read operation will return $v'$, otherwise it will return $v$.

\subsection{$i \leq v' \leq v$}

In this situation the read process will reach $br.v'$ before $br.v$, so is
guaranteed to return $v'$.

\subsection{$v' \leq i \leq v$}

As the write process will not set $br.v$ to $0$, as $v > v'$, the read process
is guaranteed to eventually reach $br.v$ and hence return $v$.

\subsubsection{Regularity of $rr$}

As such, a read operation concurrent to a write operation is guaranteed to
return either $v$ or $v'$, which makes $rr$ regular.


\section{Register emulations without correct majority?}

\end{document}
