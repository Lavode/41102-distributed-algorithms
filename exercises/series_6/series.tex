\documentclass[a4paper]{scrreprt}

% Uncomment to optimize for double-sided printing.
% \KOMAoptions{twoside}

% Set binding correction manually, if known.
% \KOMAoptions{BCOR=2cm}

% Localization options
\usepackage[english]{babel}
\usepackage[T1]{fontenc}
\usepackage[utf8]{inputenc}

% Monospaced font with support for bold
\usepackage[scaled=1.04]{couriers}

% Quotations
\usepackage{dirtytalk}

% Enhanced verbatim sections. We're mainly interested in
% \verbatiminput though.
\usepackage{verbatim}

% Automatically remove leading whitespace in lstlisting
\usepackage{lstautogobble}

% PDF-compatible landscape mode.
% Makes PDF viewers show the page rotated by 90°.
\usepackage{pdflscape}

% Advanced tables
\usepackage{array}
\usepackage{tabularx}
\usepackage{longtable}

% Fancy tablerules
\usepackage{booktabs}

% Graphics
\usepackage{graphicx}

% Current time
\usepackage[useregional=numeric]{datetime2}

% Float barriers.
% Automatically add a FloatBarrier to each \section
\usepackage[section]{placeins}

% Custom header and footer
\usepackage{fancyhdr}

\usepackage{geometry}
\usepackage{layout}

% Math tools
\usepackage{mathtools}
% Math symbols
\usepackage{amsmath,amsfonts,amssymb}
\usepackage{amsthm}

\DeclarePairedDelimiter{\ceil}{\lceil}{\rceil}
\DeclarePairedDelimiter{\floor}{\lfloor}{\rfloor}

% General symbols
\usepackage{stmaryrd}

\DeclarePairedDelimiter\abs{\lvert}{\rvert}

% Indistinguishable operator (three stacked tildes)
\newcommand*{\diffeo}{% 
  \mathrel{\vcenter{\offinterlineskip
  \hbox{$\sim$}\vskip-.35ex\hbox{$\sim$}\vskip-.35ex\hbox{$\sim$}}}}

% Bullet point
\newcommand{\tabitem}{~~\llap{\textbullet}~~}

\pagestyle{plain}
% \fancyhf{}
% \lhead{}
% \lfoot{}
% \rfoot{}
% 
% Source code & highlighting
\usepackage{listings}

% Coloured boxes!
\usepackage[most]{tcolorbox}
\newtcolorbox{library}[2][]{
	enhanced,
	sharp corners,
	coltitle=black, % title colour
	colbacktitle=black!10!white, % title bg colour
	halign title=center, % title align
	toptitle=1mm, % Top/bottom additional spacing for title
	bottomtitle=1mm,
	fonttitle=\ttfamily,
	colback=white, % body bg colour
	fontupper=\ttfamily,
	title=#2,#1
}

\newtcolorbox{boxcomment}[2][]{
	enhanced,
	colframe=white, % frame colour
	colbacktitle=white, % title bg colour
	halign=center, % body align
	colback=white, % body bg colour
	fonttitle=\ttfamily,
	fontupper=\ttfamily,
	title=#2,#1
}

% SI units
\usepackage[binary-units=true]{siunitx}
\DeclareSIUnit\cycles{cycles}

% Convenience commands
\newcommand{\mailsubject}{41100 - Distributed algorithms - Series 6}
\newcommand{\maillink}[1]{\href{mailto:#1?subject=\mailsubject}
                               {#1}}

% Should use this command wherever the print date is mentioned.
\newcommand{\printdate}{\today}

\subject{41102 - Distributed algorithms}
\title{Series 6}

\author{Michael Senn \maillink{michael.senn@students.unibe.ch} - 16-126-880}

\date{\printdate}

% Needs to be the last command in the preamble, for one reason or
% another. 
\usepackage{hyperref}


\begin{document}
\maketitle


\setcounter{chapter}{5}

\chapter{Series 6}

\section{Atomic register execution}

Let processes $\{p, q, r, s, t\}$ have ranks $\{5, 4, 3, 2, 1\}$ respectively.
In the following executions, assume messages between the processes which are not
mentioned to be delayed up to the point they are mentioned.

\subsection{Execution A}

Consider the following execution:

\begin{description}
	\item[$ts = 0$] $p$ sends $[WRITE, 1, 5, x]$ to all processes
	\item[$ts = 0$] $q$ sends $[WRITE, 1, 4, y]$ to all processes
	\item[$ts = 1$] $[WRITE, 1, 5, x]$ from $p$ arrives at $r$, is
		processed, corresponding $[ACK]$ is sent back.
	\item[$ts = 2$] $r$ starts read operation, stores $readval := x$
	\item[$ts = 2$] $r$ sends out write-back $[WRITE, 1, 5, x]$ to all
		processes
	\item[$ts = 3$] $[WRITE, 1, 4, y]$ from $q$ arrives at $s$, is
		processed, corresponding $[ACK]$ is sent back.
	\item[$ts = 4$] $s$ starts read operation, stores $readval := y$
	\item[$ts = 4$] $s$ sends out write-back $[WRITE, 1, 4, y]$ to all
		processes
	\item[$ts = 5$] All pending write-back messages are arrived and are
		processed. No matter the order, as the two timestamps of the
		writes of $p$ and $q$ are equal, all processes will now store
		$val = x, ts = 1, wr = 5$.
	\item[$ts = 5$] Read of $r$ terminates and returns $readval = x$. Read
		of $s$ terminates and returns $readval = y$.
	\item[$ts = 6$] All pending write messages by $p$ and $q$ arrive, are
		processed (being discarded as they do not contain more
		up-to-date information), corresponding $ACK$s are returned and
		processed.
	\item[$ts = 7$] Write operations of $p$ and $q$ terminate.
	\item[$ts = 8$] Non-concurrent read operation of $t$ commences as per
		the protocol, returns $readval = x$.
\end{description}

\subsection{Execution B}

Consider the following execution adapted from the one above, changes
highlighted in \textcolor{red}{red}. Specifically the messages which arrived
`early' were switched, with the message of writer $q$ arriving at $r$ early,
and the one of writer $p$ arriving at $s$ early.

\begin{description}
	\item[$ts = 0$] $p$ sends $[WRITE, 1, 5, x]$ to all processes
	\item[$ts = 0$] $q$ sends $[WRITE, 1, 4, y]$ to all processes
	\item[$ts = 1$] $\textcolor{red}{[WRITE, 1, 4, y]}$ from
		\textcolor{red}{$q$} arrives at $r$, is processed,
		corresponding $[ACK]$ is sent back.
	\item[$ts = 2$] $r$ starts read operation, stores $readval :=
		\textcolor{red}{y}$
	\item[$ts = 2$] $r$ sends out write-back $\textcolor{red}{[WRITE, 1, 4,
		y]}$ to all processes
	\item[$ts = 3$] $\textcolor{red}{[WRITE, 1, 5, x]}$ from
		$\textcolor{red}{p}$ arrives at $s$, is processed,
		corresponding $[ACK]$ is sent back.
	\item[$ts = 4$] $s$ starts read operation, stores $readval :=
		\textcolor{red}{x}$
	\item[$ts = 4$] $s$ sends out write-back $\textcolor{red}{[WRITE, 1, 5,
		x]}$ to all processes
	\item[$ts = 5$] All pending write-back messages are arrived and are
		processed. No matter the order, as the two timestamps of the
		writes of $p$ and $q$ are equal, all processes will now store
		$val = x, ts = 1, wr = 5$.
	\item[$ts = 5$] Read of $r$ terminates and returns $readval =
		\textcolor{red}{y}$. Read of $s$ terminates and returns
		$readval = \textcolor{red}{x}$.
	\item[$ts = 6$] All pending write messages by $p$ and $q$ arrive, are
		processed (being discarded as they do not contain more
		up-to-date information), corresponding $ACK$s are returned and
		processed.
	\item[$ts = 7$] Write operations of $p$ and $q$ terminate.
	\item[$ts = 8$] Non-concurrent read operation of $t$ commences as per
		the protocol, returns $readval = x$.
\end{description}

\section{Erasure-coded storage}

\subsection{Storage efficiency of $n$-node system with $f < \frac{n}{2}$ failing nodes}

As at most $f < \frac{n}{2}$ nodes will fail, any $(\ceil{\frac{n}{2}}, n)$
erasure code will be able to store data safely. As an example, consider the
following Reed-Solomon code:

Let $k := \ceil{\frac{n}{2}}$. To store a message $m$, split it into
bit-strings of length $k$. For each such string $m_i := (x_1, \cdots, x_k)$,
compute the Lagrange polynomial $p$ such that $p(i - 1) = x_i$. Then, store the
value of $p(i)$ on node number $i\ \forall\ 0 \leq i \leq n - 1$.

As at most $f < \frac{n}{2}$ nodes will fail and $f$ is an integer, it holds
that $n - f \geq \ceil{\frac{n}{2}}$, which is a sufficient number of fragments
to reconstruct $m_i$.

Such a system (or, in fact, any $(\ceil{\frac{n}{2}}, n)$ erasure code) will
have a storage efficiency of $\approx \SI{50}{\percent}$.

\subsection{Erasure-coded safe register from algorithm 4.2}

Algorithm 4.2 can be adjusted as follows, to implement an erasure-coded safe
registers. Changes to existing event handler are in \textcolor{red}{red}, with
unchanged event handlers left out for brevity. Well-known procedures such as
lagrange interpolation are assumed to be available in external libraries.

\begin{library}{Implementing one-reader n-writer safe register}
  \begin{lstlisting}[mathescape=true,autogobble=true,breaklines=true,escapeinside={<@}{@>},basicstyle=\small]
    <@\textbf{Implements}@>:
      <@\textcolor{red}{$(1, N)$ safe register, \textbf{instance} onsr}@>

    <@\textbf{upon event}@> $(onsr,\ Write\ |\ v)$ <@\textbf{do}@>:
      <@\textcolor{red}{$k := \ceil{N/2}$}@>
      <@\textcolor{red}{$(v_1 || \cdots || v_k) := v$}@>
      $wts := wts + 1$
      $acks := 0$
      // Interpolate polynomial of degree (k-1)
      <@\textcolor{red}{$p := \text{lagrange\_interpolation}((0, v_1), \cdots, (k - 1, v_k))$}@>
      <@\textcolor{red}{\textbf{for} $i := 1$ to $N$:}@>
	<@\textcolor{red}{\textbf{trigger} $(pl,\ send\ |\ p_i,\ [WRITE,\ wts,\ (i - 1,\ p(i - 1))])$}@>

    <@\textbf{upon event}@> $(pl,\ Deliver\ |\ q,\ [VALUE,\ r,\ ts',\ v'])$ <@\textbf{such that}@> $r = rid$ <@\textbf{do}@>:
      <@\textcolor{red}{$k := \ceil{N/2}$}@>
      <@\textcolor{red}{$(i, p_i) := v'$}@>
      <@\textcolor{red}{$readlist[i] := p_i$}@>
      <@\textbf{if}@> $\abs{readlist} > N/2$ <@\textbf{then}@>:
	<@\textcolor{red}{$p := \text{lagrange\_interpolation}(readlist)$}@>
        <@\textcolor{red}{\textbf{for} $i := 1$ to $k$:}@>
	  <@\textcolor{red}{$v_i := p(i- 1)$}@>
	<@\textcolor{red}{\textbf{trigger} $(onsr,\ ReadReturn\ |\ (v_1 || \cdots || v_{k})$}@>
  \end{lstlisting}
\end{library}

\subsection{Difficulties of extending this to regular semantics}

A regular register requires that a read concurrent to a write operation returns
either the value being currently written, or the most recently written value.
In the case of a erasure-coded register, however, the information of a value
$v$ is spread across $n$ different registers, with (generally )no single
register having sufficient information to reconstruct $v$ on its own.

As such information from multiple registers is needed to read $v$, which, in
the case of a distributed system, makes it hard to ensure that the set of
values read from the various registers will be consistent, that is will lead to
the reconstruction of either the current or the most recently written value.

\end{document}
