\documentclass[a4paper]{scrreprt}

% Uncomment to optimize for double-sided printing.
% \KOMAoptions{twoside}

% Set binding correction manually, if known.
% \KOMAoptions{BCOR=2cm}

% Localization options
\usepackage[english]{babel}
\usepackage[T1]{fontenc}
\usepackage[utf8]{inputenc}

% Monospaced font with support for bold
\usepackage[scaled=1.04]{couriers}

% Quotations
\usepackage{dirtytalk}

% Enhanced verbatim sections. We're mainly interested in
% \verbatiminput though.
\usepackage{verbatim}

% Automatically remove leading whitespace in lstlisting
\usepackage{lstautogobble}

% PDF-compatible landscape mode.
% Makes PDF viewers show the page rotated by 90°.
\usepackage{pdflscape}

% Advanced tables
\usepackage{array}
\usepackage{tabularx}
\usepackage{longtable}

% Fancy tablerules
\usepackage{booktabs}

% Graphics
\usepackage{graphicx}

% Current time
\usepackage[useregional=numeric]{datetime2}

% Float barriers.
% Automatically add a FloatBarrier to each \section
\usepackage[section]{placeins}

% Custom header and footer
\usepackage{fancyhdr}

\usepackage{geometry}
\usepackage{layout}

% Math tools
\usepackage{mathtools}
% Math symbols
\usepackage{amsmath,amsfonts,amssymb}
\usepackage{amsthm}

\DeclarePairedDelimiter{\ceil}{\lceil}{\rceil}
\DeclarePairedDelimiter{\floor}{\lfloor}{\rfloor}

% General symbols
\usepackage{stmaryrd}

\DeclarePairedDelimiter\abs{\lvert}{\rvert}

% Indistinguishable operator (three stacked tildes)
\newcommand*{\diffeo}{% 
  \mathrel{\vcenter{\offinterlineskip
  \hbox{$\sim$}\vskip-.35ex\hbox{$\sim$}\vskip-.35ex\hbox{$\sim$}}}}

% Bullet point
\newcommand{\tabitem}{~~\llap{\textbullet}~~}

\pagestyle{plain}
% \fancyhf{}
% \lhead{}
% \lfoot{}
% \rfoot{}
% 
% Source code & highlighting
\usepackage{listings}

% Coloured boxes!
\usepackage[most]{tcolorbox}
\newtcolorbox{library}[2][]{
	enhanced,
	sharp corners,
	coltitle=black, % title colour
	colbacktitle=black!10!white, % title bg colour
	halign title=center, % title align
	toptitle=1mm, % Top/bottom additional spacing for title
	bottomtitle=1mm,
	fonttitle=\ttfamily,
	colback=white, % body bg colour
	fontupper=\ttfamily,
	title=#2,#1
}

\newtcolorbox{boxcomment}[2][]{
	enhanced,
	colframe=white, % frame colour
	colbacktitle=white, % title bg colour
	halign=center, % body align
	colback=white, % body bg colour
	fonttitle=\ttfamily,
	fontupper=\ttfamily,
	title=#2,#1
}

% SI units
\usepackage[binary-units=true]{siunitx}
\DeclareSIUnit\cycles{cycles}

% Convenience commands
\newcommand{\mailsubject}{41100 - Distributed algorithms - Series 6}
\newcommand{\maillink}[1]{\href{mailto:#1?subject=\mailsubject}
                               {#1}}

% Should use this command wherever the print date is mentioned.
\newcommand{\printdate}{\today}

\subject{41102 - Distributed algorithms}
\title{Series 6}

\author{Michael Senn \maillink{michael.senn@students.unibe.ch} - 16-126-880}

\date{\printdate}

% Needs to be the last command in the preamble, for one reason or
% another. 
\usepackage{hyperref}


\begin{document}
\maketitle


\setcounter{chapter}{5}

\chapter{Series 6}

\section{Atomic register execution}

Let processes $\{p, q, r, s, t\}$ have ranks $\{5, 4, 3, 2, 1\}$ respectively.
In the following executions, assume messages between the processes which are not
mentioned to be delayed up to the point they are mentioned.

\subsection{Execution A}

Consider the following execution:

\begin{description}
	\item[$ts = 0$] $p$ sends $[WRITE, 1, 5, x]$ to all processes
	\item[$ts = 0$] $q$ sends $[WRITE, 1, 4, y]$ to all processes
	\item[$ts = 1$] $[WRITE, 1, 5, x]$ from $p$ arrives at $r$, is
		processed, corresponding $[ACK]$ is sent back.
	\item[$ts = 2$] $r$ starts read operation, stores $readval := x$
	\item[$ts = 2$] $r$ sends out write-back $[WRITE, 1, 5, x]$ to all
		processes
	\item[$ts = 3$] $[WRITE, 1, 4, y]$ from $q$ arrives at $s$, is
		processed, corresponding $[ACK]$ is sent back.
	\item[$ts = 4$] $s$ starts read operation, stores $readval := y$
	\item[$ts = 4$] $s$ sends out write-back $[WRITE, 1, 4, y]$ to all
		processes
	\item[$ts = 5$] All pending write-back messages are arrived and are
		processed. No matter the order, as the two timestamps of the
		writes of $p$ and $q$ are equal, all processes will now store
		$val = x, ts = 1, wr = 5$.
	\item[$ts = 5$] Read of $r$ terminates and returns $readval = x$. Read
		of $s$ terminates and returns $readval = y$.
	\item[$ts = 6$] All pending write messages by $p$ and $q$ arrive, are
		processed (being discarded as they do not contain more
		up-to-date information), corresponding $ACK$s are returned and
		processed.
	\item[$ts = 7$] Write operations of $p$ and $q$ terminate.
	\item[$ts = 8$] Non-concurrent read operation of $t$ commences as per
		the protocol, returns $readval = x$.
\end{description}

\subsection{Execution B}

Consider the following execution adapted from the one above, changes
highlighted in \textcolor{red}{red}. Specifically the messages which arrived
`early' were switched, with the message of writer $q$ arriving at $r$ early,
and the one of writer $p$ arriving at $s$ early.

\begin{description}
	\item[$ts = 0$] $p$ sends $[WRITE, 1, 5, x]$ to all processes
	\item[$ts = 0$] $q$ sends $[WRITE, 1, 4, y]$ to all processes
	\item[$ts = 1$] $\textcolor{red}{[WRITE, 1, 4, y]}$ from
		\textcolor{red}{$q$} arrives at $r$, is processed,
		corresponding $[ACK]$ is sent back.
	\item[$ts = 2$] $r$ starts read operation, stores $readval :=
		\textcolor{red}{y}$
	\item[$ts = 2$] $r$ sends out write-back $\textcolor{red}{[WRITE, 1, 4,
		y]}$ to all processes
	\item[$ts = 3$] $\textcolor{red}{[WRITE, 1, 5, x]}$ from
		$\textcolor{red}{p}$ arrives at $s$, is processed,
		corresponding $[ACK]$ is sent back.
	\item[$ts = 4$] $s$ starts read operation, stores $readval :=
		\textcolor{red}{x}$
	\item[$ts = 4$] $s$ sends out write-back $\textcolor{red}{[WRITE, 1, 5,
		x]}$ to all processes
	\item[$ts = 5$] All pending write-back messages are arrived and are
		processed. No matter the order, as the two timestamps of the
		writes of $p$ and $q$ are equal, all processes will now store
		$val = x, ts = 1, wr = 5$.
	\item[$ts = 5$] Read of $r$ terminates and returns $readval =
		\textcolor{red}{y}$. Read of $s$ terminates and returns
		$readval = \textcolor{red}{x}$.
	\item[$ts = 6$] All pending write messages by $p$ and $q$ arrive, are
		processed (being discarded as they do not contain more
		up-to-date information), corresponding $ACK$s are returned and
		processed.
	\item[$ts = 7$] Write operations of $p$ and $q$ terminate.
	\item[$ts = 8$] Non-concurrent read operation of $t$ commences as per
		the protocol, returns $readval = x$.
\end{description}

\section{Erasure-coded storage}

\end{document}
