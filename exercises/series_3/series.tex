\documentclass[a4paper]{scrreprt}

% Uncomment to optimize for double-sided printing.
% \KOMAoptions{twoside}

% Set binding correction manually, if known.
% \KOMAoptions{BCOR=2cm}

% Localization options
\usepackage[english]{babel}
\usepackage[T1]{fontenc}
\usepackage[utf8]{inputenc}

% Quotations
\usepackage{dirtytalk}

% Enhanced verbatim sections. We're mainly interested in
% \verbatiminput though.
\usepackage{verbatim}

% Automatically remove leading whitespace in lstlisting
\usepackage{lstautogobble}

% PDF-compatible landscape mode.
% Makes PDF viewers show the page rotated by 90°.
\usepackage{pdflscape}

% Advanced tables
\usepackage{array}
\usepackage{tabularx}
\usepackage{longtable}

% Fancy tablerules
\usepackage{booktabs}

% Graphics
\usepackage{graphicx}

% Current time
\usepackage[useregional=numeric]{datetime2}

% Float barriers.
% Automatically add a FloatBarrier to each \section
\usepackage[section]{placeins}

% Custom header and footer
\usepackage{fancyhdr}

\usepackage{geometry}
\usepackage{layout}

% Math tools
\usepackage{mathtools}
% Math symbols
\usepackage{amsmath,amsfonts,amssymb}
\usepackage{amsthm}
% General symbols
\usepackage{stmaryrd}

\DeclarePairedDelimiter\abs{\lvert}{\rvert}

% Indistinguishable operator (three stacked tildes)
\newcommand*{\diffeo}{% 
  \mathrel{\vcenter{\offinterlineskip
  \hbox{$\sim$}\vskip-.35ex\hbox{$\sim$}\vskip-.35ex\hbox{$\sim$}}}}

% Bullet point
\newcommand{\tabitem}{~~\llap{\textbullet}~~}

\pagestyle{plain}
% \fancyhf{}
% \lhead{}
% \lfoot{}
% \rfoot{}
% 
% Source code & highlighting
\usepackage{listings}

% Coloured boxes!
\usepackage[most]{tcolorbox}
\newtcolorbox{library}[2][]{
	enhanced,
	sharp corners,
	coltitle=black, % title colour
	colbacktitle=black!10!white, % title bg colour
	halign title=center, % title align
	toptitle=1mm, % Top/bottom additional spacing for title
	bottomtitle=1mm,
	fonttitle=\ttfamily,
	colback=white, % body bg colour
	fontupper=\ttfamily,
	title=#2,#1
}

\newtcolorbox{boxcomment}[2][]{
	enhanced,
	colframe=white, % frame colour
	colbacktitle=white, % title bg colour
	halign=center, % body align
	colback=white, % body bg colour
	fonttitle=\ttfamily,
	fontupper=\ttfamily,
	title=#2,#1
}

% SI units
\usepackage[binary-units=true]{siunitx}
\DeclareSIUnit\cycles{cycles}

% Convenience commands
\newcommand{\mailsubject}{41100 - Distributed algorithms - Series 3}
\newcommand{\maillink}[1]{\href{mailto:#1?subject=\mailsubject}
                               {#1}}

% Should use this command wherever the print date is mentioned.
\newcommand{\printdate}{\today}

\subject{41102 - Distributed algorithms}
\title{Series 3}

\author{Michael Senn \maillink{michael.senn@students.unibe.ch} - 16-126-880}

\date{\printdate}

% Needs to be the last command in the preamble, for one reason or
% another. 
\usepackage{hyperref}


\begin{document}
\maketitle


\setcounter{chapter}{2}

\chapter{Series 3}

\section{Relations among failure detectors}

Given a non-perfect failure detector $NP$, with strong accuracy and weak
completness, a perfect failure detector $P$ can be implemented as follows. This
utilizes the fact that, if a process $p$ running $NP$ detects a process $p'$ to
have crashed, due to the strong accuracy property of $NP$, $p'$ is guaranteed
to actually have crashed.

\begin{library}{Implementing $P$ using $NP$}
  \begin{lstlisting}[mathescape=true,autogobble=true,breaklines=true]
    UPON init:
      alive := $\Pi$
      detected := {}

    UPON <NP.crash | p>:
      $\text{detected} := $\text{detected} \cup p$
      $\text{alive} := $\text{alive} \cup \{p\}$

      SEND [p crashed] to all $p' \in \Pi$

    UPON receive [p crashed] from $p'$:
      $\text{detected} := $\text{detected} \cup p$
      $\text{alive} := $\text{alive} \cup p$
  \end{lstlisting}
\end{library}

\section{Perfect Failure Detector}

For conciseness we assume that we have access to a slightly extended timer
object, with the following properties:
\begin{itemize}
  \item Each timer carries a reference to a process $p$, which is emitted as a
    parameter in its timeout event.
  \item Each such timer can be acessed via its parameter $p$.
  \item Each timer can be reset - that is resetting its countdown to the value
    it was started with.
\end{itemize}

With this, a perfect failure detector based on unidirectional heartbeats can be
implemented as follows, with an interface identicaly to the one in the lecture.

For conciseness we assume that we have access to a timer which can be reset - resetting its countdown to the given value - as well as

\begin{library}{Implementing $P$ using heartbeats}
  \begin{lstlisting}[mathescape=true,autogobble=true,breaklines=true]
    UPON init:
      alive := $\Pi$
      detected := {}

      for $p \in \Pi$:
        // Start timer tracking lifetime of $p$
        start_timer($\Delta + \Phi$, p)

    UPON <timeout | p>:
      $\text{detected} := $\text{detected} \cup p$
      $\text{alive} := $\text{alive} \cup \{p\}$

      TRIGGER <P.Crash | p>

    UPON receive [HEARTBEAT] from p:
      // Restart timer tracking lifetime of $p$
      restart_timer(p)
  \end{lstlisting}
\end{library}

\section{Quorum systems}


\end{document}
