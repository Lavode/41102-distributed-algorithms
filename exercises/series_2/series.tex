\documentclass[a4paper]{scrreprt}

% Uncomment to optimize for double-sided printing.
% \KOMAoptions{twoside}

% Set binding correction manually, if known.
% \KOMAoptions{BCOR=2cm}

% Localization options
\usepackage[english]{babel}
\usepackage[T1]{fontenc}
\usepackage[utf8]{inputenc}

% Quotations
\usepackage{dirtytalk}

% Enhanced verbatim sections. We're mainly interested in
% \verbatiminput though.
\usepackage{verbatim}

% Automatically remove leading whitespace in lstlisting
\usepackage{lstautogobble}

% PDF-compatible landscape mode.
% Makes PDF viewers show the page rotated by 90°.
\usepackage{pdflscape}

% Advanced tables
\usepackage{array}
\usepackage{tabularx}
\usepackage{longtable}

% Fancy tablerules
\usepackage{booktabs}

% Graphics
\usepackage{graphicx}

% Current time
\usepackage[useregional=numeric]{datetime2}

% Float barriers.
% Automatically add a FloatBarrier to each \section
\usepackage[section]{placeins}

% Custom header and footer
\usepackage{fancyhdr}

\usepackage{geometry}
\usepackage{layout}

% Math tools
\usepackage{mathtools}
% Math symbols
\usepackage{amsmath,amsfonts,amssymb}
\usepackage{amsthm}
% General symbols
\usepackage{stmaryrd}

\DeclarePairedDelimiter\abs{\lvert}{\rvert}

% Indistinguishable operator (three stacked tildes)
\newcommand*{\diffeo}{% 
  \mathrel{\vcenter{\offinterlineskip
  \hbox{$\sim$}\vskip-.35ex\hbox{$\sim$}\vskip-.35ex\hbox{$\sim$}}}}

% Bullet point
\newcommand{\tabitem}{~~\llap{\textbullet}~~}

\pagestyle{plain}
% \fancyhf{}
% \lhead{}
% \lfoot{}
% \rfoot{}
% 
% Source code & highlighting
\usepackage{listings}

% Coloured boxes!
\usepackage[most]{tcolorbox}
\newtcolorbox{library}[2][]{
	enhanced,
	sharp corners,
	coltitle=black, % title colour
	colbacktitle=black!10!white, % title bg colour
	halign title=center, % title align
	toptitle=1mm, % Top/bottom additional spacing for title
	bottomtitle=1mm,
	fonttitle=\ttfamily,
	colback=white, % body bg colour
	fontupper=\ttfamily,
	title=#2,#1
}

\newtcolorbox{boxcomment}[2][]{
	enhanced,
	colframe=white, % frame colour
	colbacktitle=white, % title bg colour
	halign=center, % body align
	colback=white, % body bg colour
	fonttitle=\ttfamily,
	fontupper=\ttfamily,
	title=#2,#1
}

% SI units
\usepackage[binary-units=true]{siunitx}
\DeclareSIUnit\cycles{cycles}

% Convenience commands
\newcommand{\mailsubject}{41100 - Distributed algorithms - Series 2}
\newcommand{\maillink}[1]{\href{mailto:#1?subject=\mailsubject}
                               {#1}}

% Should use this command wherever the print date is mentioned.
\newcommand{\printdate}{\today}

\subject{41102 - Distributed algorithms}
\title{Series 2}

\author{Michael Senn \maillink{michael.senn@students.unibe.ch} - 16-126-880}

\date{\printdate}

% Needs to be the last command in the preamble, for one reason or
% another. 
\usepackage{hyperref}


\begin{document}
\maketitle


\setcounter{chapter}{1}

\chapter{Series 2}

\section{Pizza or pasta?}

Given the following assumptions:
\begin{itemize}
	\item Every customer places at most one order at a time - for either
		pizza or pasta.
	\item The communication channel between customers, Alice, Bob and Carol
		is a perfect link
	\item Our algorithms have access to basic data structures such as hash
		maps and queues.
\end{itemize}

Then the following modules and algorithms describe an implementation providing
the requested properties:

\begin{itemize}
	\item Every customer will get his/her order delivered
	\item No dish is delivered if it was not ordered
	\item Every customer gets each order exactly once
\end{itemize}

\begin{library}{Module Carol $c$}
        \begin{lstlisting}[mathescape=true,autogobble=true,breaklines=true]
		Requests:
		  <c.prepare | customer>: Start preparing pasta for the given customer
		Indications:
		  <c.ready | customer>: Pasta for given customer is ready
		  <c.reject | customer>: Unable to prepare pasta for given customer as queue is full
		Properties:
		  Every submitted order is eventually rejected or finished
		  If an order was placed it is prepared exactly once
		  Only orders which are placed will be prepared

		UPON init:
		  # FIFO queue of fixed size, supports .push, .pop and .size
		  queue := Queue(size = 7)

		UPON <c.prepare | customer>:
		  if queue.size == 3:
		    TRIGGER <c.reject | customer>
		  else
		    queue.push(customer)

		UPON queue.size != 0:
		  customer = queue.pop
		  prepare(customer)
		  TRIGGER <c.ready | customer>
        \end{lstlisting}
\end{library}

\begin{library}{Module Bob $b$}
        \begin{lstlisting}[mathescape=true,autogobble=true,breaklines=true]
		Requests:
		  <b.prepare | customer>: Start preparing a pizza for the given customer
		Indications:
		  <b.ready | customer>: Pizza for given customer is ready
		  <b.reject | customer>: Unable to prepare pizza for given customer as queue is full
		Properties:
		  Every submitted order is eventually rejected or finished
		  If an order was placed it is prepared exactly once
		  Only orders which are placed will be prepared

		UPON init:
		  # FIFO queue of fixed size, supports .push, .pop and .size
		  queue := Queue(size = 3)

		UPON <b.prepare | customer>:
		  if queue.size == 3:
		    TRIGGER <b.reject | customer>
		  else
		    queue.push(customer)

		UPON queue.size != 0:
		  customer = queue.pop
		  prepare(customer)
		  TRIGGER <b.ready | customer>
        \end{lstlisting}
\end{library}

\begin{library}{Module Alice $a$}
        \begin{lstlisting}[mathescape=true,autogobble=true,breaklines=true]
		Requests:
		  <a.order | customer, dish>: Place order for given dish by given customer
		  
		Indications:
		  <a.serve | customer, dish>: Serve given dish to given customer

		Properties:
		  Every submitted order is eventually served exactly once
		  No order is served unless it was submitted

		UPON init:
		  # FIFO queue of unbounded size
		  queue := Queue()

		UPON <a.order | customer, dish>:
		  if dish == pizza:
		    TRIGGER <b.prepare | customer>
		  elsif dish == pasta:
		    TRIGGER <c.prepare | customer>

		UPON <b.reject | customer>:
		  queue.push((customer, pizza))

		UPON <c.reject | customer>:
		  queue.push((customer, pasta))

		UPON queue.size != 0:
		  (customer, dish) := queue.pop
		  TRIGGER <a.order | customer, dish>

		UPON <b.ready | customer>:
		  TRIGGER <a.serve | customer, pizza>

		UPON <c.ready | customer>:
		  TRIGGER <a.serve | customer, pasta>
        \end{lstlisting}
\end{library}

\section{Safety and liveness}

\section{Unreliable clocks}

\end{document}
