\documentclass[a4paper]{scrreprt}

% Uncomment to optimize for double-sided printing.
% \KOMAoptions{twoside}

% Set binding correction manually, if known.
% \KOMAoptions{BCOR=2cm}

% Localization options
\usepackage[english]{babel}
\usepackage[T1]{fontenc}
\usepackage[utf8]{inputenc}

% Monospaced font with support for bold
\usepackage[scaled=1.04]{couriers}

% Quotations
\usepackage{dirtytalk}

% Enhanced verbatim sections. We're mainly interested in
% \verbatiminput though.
\usepackage{verbatim}

% Automatically remove leading whitespace in lstlisting
\usepackage{lstautogobble}

% PDF-compatible landscape mode.
% Makes PDF viewers show the page rotated by 90°.
\usepackage{pdflscape}

% Advanced tables
\usepackage{array}
\usepackage{tabularx}
\usepackage{longtable}

% Fancy tablerules
\usepackage{booktabs}

% Graphics
\usepackage{graphicx}

% Current time
\usepackage[useregional=numeric]{datetime2}

% Float barriers.
% Automatically add a FloatBarrier to each \section
\usepackage[section]{placeins}

% Custom header and footer
\usepackage{fancyhdr}

\usepackage{geometry}
\usepackage{layout}

% Math tools
\usepackage{mathtools}
% Math symbols
\usepackage{amsmath,amsfonts,amssymb}
\usepackage{amsthm}

\DeclarePairedDelimiter{\ceil}{\lceil}{\rceil}
\DeclarePairedDelimiter{\floor}{\lfloor}{\rfloor}

% General symbols
\usepackage{stmaryrd}

\DeclarePairedDelimiter\abs{\lvert}{\rvert}

% Indistinguishable operator (three stacked tildes)
\newcommand*{\diffeo}{% 
  \mathrel{\vcenter{\offinterlineskip
  \hbox{$\sim$}\vskip-.35ex\hbox{$\sim$}\vskip-.35ex\hbox{$\sim$}}}}

% Bullet point
\newcommand{\tabitem}{~~\llap{\textbullet}~~}

\pagestyle{plain}
% \fancyhf{}
% \lhead{}
% \lfoot{}
% \rfoot{}
% 
% Source code & highlighting
\usepackage{listings}

% Coloured boxes!
\usepackage[most]{tcolorbox}
\newtcolorbox{library}[2][]{
	enhanced,
	sharp corners,
	coltitle=black, % title colour
	colbacktitle=black!10!white, % title bg colour
	halign title=center, % title align
	toptitle=1mm, % Top/bottom additional spacing for title
	bottomtitle=1mm,
	fonttitle=\ttfamily,
	colback=white, % body bg colour
	fontupper=\ttfamily,
	title=#2,#1
}

\newtcolorbox{boxcomment}[2][]{
	enhanced,
	colframe=white, % frame colour
	colbacktitle=white, % title bg colour
	halign=center, % body align
	colback=white, % body bg colour
	fonttitle=\ttfamily,
	fontupper=\ttfamily,
	title=#2,#1
}

% SI units
\usepackage[binary-units=true]{siunitx}
\DeclareSIUnit\cycles{cycles}

% Convenience commands
\newcommand{\mailsubject}{41100 - Distributed algorithms - Series 8}
\newcommand{\maillink}[1]{\href{mailto:#1?subject=\mailsubject}
                               {#1}}

% Should use this command wherever the print date is mentioned.
\newcommand{\printdate}{\today}

\subject{41102 - Distributed algorithms}
\title{Series 8}

\author{Michael Senn \maillink{michael.senn@students.unibe.ch} - 16-126-880}

\date{\printdate}

% Needs to be the last command in the preamble, for one reason or
% another. 
\usepackage{hyperref}


\begin{document}
\maketitle


\setcounter{chapter}{7}

\chapter{Series 8}

\section{Total-order broadcast using consensus}

\subsection{Importance of deterministic ordering on consensus set}

The consensus will decide on a set of messages which to deliver. A set in the
mathematical sense, as well as in many practical implementations, has no
deterministic ordering of its elements. As such, while all processes will
deliver the same messages, they might not all deliver it in the same order.
This violates the total order property of a total-order broadcast.

\subsection{Implementation without deterministic ordering}

A total-order broadcast without the use of a deterministic sorting of payload
messages can be achieved by changing the implementation such that each process
proposes a single message to the consensus. The consensus will then agree on
delivering one message at a time, which will guarantee that all processes
deliver the messages in the same sequence.

\section{Atomic register as a replicated state machine}

\begin{library}{Implementing (N N) atomic register using total-order broadcast}
  \begin{lstlisting}[mathescape=true,autogobble=true,breaklines=true,escapeinside={<@}{@>},basicstyle=\small]
    <@\textbf{Implements}@>:
      $(N, N)$ atomic register, <@\textbf{instance} nnar@>

    <@\textbf{Uses}@>:
      Total-order broadcast, <@\textbf{instance} tob@>

    <@\textbf{upon init} \textbf{do}@>:
      $val := \bot$

    <@\textbf{upon event}@> $(nnar,\ Write\ |\ v)$ <@\textbf{do}@>:
      <@\textbf{trigger} $(tob,\ Broadcast\ |\ [WRITE,\ val])$@>

    <@\textbf{upon event}@> $(tob,\ Deliver\ |\ p,\ [WRITE,\ val'])$ <@\textbf{do}@>:
      $val := val'$
      if $p = self$:
        <@\textbf{trigger} $(nnar,\ WriteReturn)$@>

    <@\textbf{upon event}@> $(nnar,\ Read)$ <@\textbf{do}@>:
      <@\textbf{trigger} $(tob,\ Broadcast\ |\ [READ])$@>

    <@\textbf{upon event}@> $(tob,\ Deliver\ |\ p,\ [READ])$ <@\textbf{do}@>:
      if $p = self$:
        <@\textbf{trigger} $(nnar,\ ReadReturn\ |\ val)$@>
      
  \end{lstlisting}
\end{library}

The termination property follows directly from the validity and agreement
properties of the total-order broadcast, as any message broadcast by a correct
process will eventually be delivered by itself, as well as any other correct
process. Hence both the $WRITE$ as well as the $READ$ messages will arrive
eventually, making the implementation terminate.

The atomicity follows from the property of a total-order broadcasts that
messages can be linearized to the same linear sequence across multiple
processes - in other words that the eventual sequence of delivered messages
will be the same for each process.

As such, when a process delivers its $READ$ message $r_1$ and then returns the
locally stored value, it is guaranteed that the result of any subsequent read
operation - which comes with its own $READ$ message $r_2$ - will have that
$r_2$ delivered after $r_1$, ensuring that the returned value will be equal or
newer.

\section{Replicated register with local read}

As total-order broadcast only guarantees that, for any process, the history of
delivered messages $(m_1 \cdots m_{n-1})$ when delivering a message $m_n$ will
be equal, but does not provide any guarantees about what the message history of
other processes is now, or will be in the future, a local read might break
atomicity. As an example consider the following execution with two processes,
$p$ and $q$.

\begin{itemize}
	\item $p$ nnar-writes $x$, broadcasts $[WRITE, x]$
	\item $p$ tob-delivers $[WRITE, x]$, sets $val := x$, nnar-write-returns
	\item $p$ nnar-reads, reads locally, returns $val = x$
	\item $q$ nnar-reads, reads locally, returns $val = \bot$
	\item $q$ tob-delivers  $[WRITE, x]$, sets $val := x$
\end{itemize}

The total-order property holds, as the tob-delivered messages of both processes
are $([WRITE, x])$. As a result both state machines are in sync, and yet the
atomicity property of the regular register is violated as the second read
returned an earlier value than a read preceeding it.

\end{document}
