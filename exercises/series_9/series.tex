\documentclass[a4paper]{scrreprt}

% Uncomment to optimize for double-sided printing.
% \KOMAoptions{twoside}

% Set binding correction manually, if known.
% \KOMAoptions{BCOR=2cm}

% Localization options
\usepackage[english]{babel}
\usepackage[T1]{fontenc}
\usepackage[utf8]{inputenc}

% Monospaced font with support for bold
\usepackage[scaled=1.04]{couriers}

% Quotations
\usepackage{dirtytalk}

% Enhanced verbatim sections. We're mainly interested in
% \verbatiminput though.
\usepackage{verbatim}

% Automatically remove leading whitespace in lstlisting
\usepackage{lstautogobble}

% PDF-compatible landscape mode.
% Makes PDF viewers show the page rotated by 90°.
\usepackage{pdflscape}

% Advanced tables
\usepackage{array}
\usepackage{tabularx}
\usepackage{longtable}

% Fancy tablerules
\usepackage{booktabs}

% Graphics
\usepackage{graphicx}

% Current time
\usepackage[useregional=numeric]{datetime2}

% Float barriers.
% Automatically add a FloatBarrier to each \section
\usepackage[section]{placeins}

% Custom header and footer
\usepackage{fancyhdr}

\usepackage{geometry}
\usepackage{layout}

% Math tools
\usepackage{mathtools}
% Math symbols
\usepackage{amsmath,amsfonts,amssymb}
\usepackage{amsthm}

\DeclarePairedDelimiter{\ceil}{\lceil}{\rceil}
\DeclarePairedDelimiter{\floor}{\lfloor}{\rfloor}

% General symbols
\usepackage{stmaryrd}

\DeclarePairedDelimiter\abs{\lvert}{\rvert}

% Indistinguishable operator (three stacked tildes)
\newcommand*{\diffeo}{% 
  \mathrel{\vcenter{\offinterlineskip
  \hbox{$\sim$}\vskip-.35ex\hbox{$\sim$}\vskip-.35ex\hbox{$\sim$}}}}

% Bullet point
\newcommand{\tabitem}{~~\llap{\textbullet}~~}

\pagestyle{plain}
% \fancyhf{}
% \lhead{}
% \lfoot{}
% \rfoot{}
% 
% Source code & highlighting
\usepackage{listings}

% Coloured boxes!
\usepackage[most]{tcolorbox}
\newtcolorbox{library}[2][]{
	enhanced,
	sharp corners,
	coltitle=black, % title colour
	colbacktitle=black!10!white, % title bg colour
	halign title=center, % title align
	toptitle=1mm, % Top/bottom additional spacing for title
	bottomtitle=1mm,
	fonttitle=\ttfamily,
	colback=white, % body bg colour
	fontupper=\ttfamily,
	title=#2,#1
}

\newtcolorbox{boxcomment}[2][]{
	enhanced,
	colframe=white, % frame colour
	colbacktitle=white, % title bg colour
	halign=center, % body align
	colback=white, % body bg colour
	fonttitle=\ttfamily,
	fontupper=\ttfamily,
	title=#2,#1
}

% SI units
\usepackage[binary-units=true]{siunitx}
\DeclareSIUnit\cycles{cycles}

% Convenience commands
\newcommand{\mailsubject}{41100 - Distributed algorithms - Series 9}
\newcommand{\maillink}[1]{\href{mailto:#1?subject=\mailsubject}
                               {#1}}

% Should use this command wherever the print date is mentioned.
\newcommand{\printdate}{\today}

\subject{41102 - Distributed algorithms}
\title{Series 9}

\author{Michael Senn \maillink{michael.senn@students.unibe.ch} - 16-126-880}

\date{\printdate}

% Needs to be the last command in the preamble, for one reason or
% another. 
\usepackage{hyperref}


\begin{document}
\maketitle


\setcounter{chapter}{8}

\chapter{Series 9}

\section{Flooding Consensus}

Consider a system with two processes $p$, $q$.

\subsection{Reducing number of rounds for one process in two-process system}

As $N = 2$, reducing the number of rounds for one process requires lowering the
number of rounds for either $p$ or $q$ to at least $N - 1 = 1$. Consider then
the following execution where the number of rounds of $p$ is lowered to $1$.

\begin{itemize}
	\item $p$ proposes $x$, broadcasts $[PROPOSAL, 1, \{x\}]$
	\item $q$ proposes $y$, broadcasts $[PROPOSAL, 1, \{y\}]$
	\item $p$ beb-delivers $[PROPOSAL, 1, \{x\}]$ and $[PROPOSAL, 1,
		\{y\}]$.
	\item As $p$ has received messages from all healthy processes, and $r =
		N - 1 = 1$ it decides $x$.
	\item $p$ crashes. $q$ detects $p$ as having crashed.
	\item $q$ beb-delivers $[PROPOSAL, 1, \{y\}]$. As it has received
		messages from all healthy processes, it proceeds to round $r =
		2$.
	\item $q$ broadcast and beb-delivers $[PROPOSAL, 2, \{y\}]$.
	\item As $q$ has received messages from all healthy processes, and $r =
		N = 2$, it decides $y$.
\end{itemize}

This violates the uniform agreement property, as two processes have decided
differently. The argument for lowering the rounds for $q$ follows from
symmetry. Hence it is not possible to lower the rounds of either process.

\subsection{Correctness with eventually-perfect failure detector}

Consider the following execution.

\begin{itemize}
	\item $p$ proposes $x$, broadcasts $[PROPOSAL, 1, \{x\}]$
	\item $q$ proposes $y$, broadcasts $[PROPOSAL, 1, \{y\}]$
	\item $p$ beb-delivers both messages, proceeds to round $r = 2$, broadcasts $[PROPOSAL, 2, \{x, y\}]$
	\item $q$ suspects $p$ of having crashed
	\item $q$ beb-delivers $[PROPOSAL, 1, \{y\}]$, proceeds to round $r = 2$, broadcasts $[PROPOSAL, 2, \{y\}]$
	\item $p$ beb-delivers $[PROPOSAL, 2, \{x, y\}]$ and $[PROPOSAL, 2, \{y\}]$, decides $x$
	\item $q$ beb-delivers $[PROPOSAL, 2, \{y\}]$, decides $y$.
\end{itemize}

This violates the uniform agreement property once more, as two processes have
decided on different values. Hence the given algorithm does not work with an
eventually-perfect failure detector, as excluding correct processes which are
suspected of having crashed can cause issues as shown.

\section{Leader-Driven Consensus}

\subsection{Requirement for majority of correct processes}

The requirement for a majority of correct processes stems from the need to
establish a quorum. As one requirement for a quorum is that any two quorum must
overlap in at least one process, the size of a quorum must be greater than $N /
2$ - which can only be guaranteed to be achieved if there is a majority of
correct processes.

\subsection{Violations if lacking majority of correct processes}

If there is no majority of correct processes yet the quorum size remains
untouched, then the termination process will be violated as the leader will be
waiting - indefinitely - for sufficiently many responses from correct
processes.

If the `quorum' size was lowered such that the leader only waits for however
many correct processes are available, then the resulting sets of processes
would not be quorums (as they might not necessarily overlap in at least one
process), which then might violate the lock-in property of the epoch consensus
and, subsequently, the uniform agreement property of the uniform consensus.

\subsection{Example}

As an example for violating the uniform agreement property if the `quorum' size
is relaxed, consider a system with four processes $p, q, r, s$ with ranks in
order $1, 2, 3, 4$. Assume that we relax the requirement (and hence the
`quorum' size) to $F = N/2$, so no majority of correct processes is required.
Consider then the following execution.

\begin{itemize}
	\item An epoch starts for all processes, with $p$ as a leader and epoch
		timestamp $1$
	\item $p$ proposes $x$
	\item $p$ broadcasts a read
	\item $p$ and $q$ respond with their state $(0, \bot)$
	\item As neither process knows of a previously-decided value, $p$
		broadcasts a write for value $x$
	\item $p$ and $q$ store $(1, x)$ and respond with an accept
	\item $p$ receives the two accepts, and broadcasts a decide for value
		$x$
	\item $p$ and $q$ receive the message and decide $x$
	\item $p$ and $q$ crash
	\item A new epoch starts, with $r$ as leader and epoch timestamp $3$
	\item $r$ proposes $z$.
	\item $r$ broadcasts a read
	\item $r$ and $s$ respond with their state $(0, \bot)$
	\item As neither process knows of a previously-decided value, $r$
		broadcasts a write for value $z$
	\item $r$ and $z$ store $(3, z)$ and respond with an accept
	\item $r$ receives the two accepts, and broadcasts a decide for value
		$z$
	\item $r$ and $s$ receive the message and decide $z$
\end{itemize}

This leads to $p$ and $q$ deciding $x$, and $r$ and $s$ deciding $z$, violating
the uniform agreement property.

As an example for violating the termination property if the quorum size remains
untouched, simply consider the case where $r$ and $s$ have crashed in the
beginning. The leader $p$ will then, after having broadcast a READ, wait for 3
STATE messages - a requirement which will never be satisfied as only 2 correct
processes remain.

\section{Leader-Driven Consensus, optimised}

The READ and STATE messages in the epoch consensus serve to establish the
lock-in property. Specifically, if a leader has decided in an earlier epoch, the
current epoch's leader must decide the same value.

In the case of the first epoch consensus there cannot have been an earlier
decision, as there was no earlier epoch. All processes are guaranteed to still
have their state set to what they were initialised with, so this set of
messages can be left out.

\end{document}
