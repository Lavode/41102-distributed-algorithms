\documentclass[a4paper]{scrreprt}

% Uncomment to optimize for double-sided printing.
% \KOMAoptions{twoside}

% Set binding correction manually, if known.
% \KOMAoptions{BCOR=2cm}

% Localization options
\usepackage[english]{babel}
\usepackage[T1]{fontenc}
\usepackage[utf8]{inputenc}

% Monospaced font with support for bold
\usepackage[scaled=1.04]{couriers}

% Quotations
\usepackage{dirtytalk}

% Enhanced verbatim sections. We're mainly interested in
% \verbatiminput though.
\usepackage{verbatim}

% Automatically remove leading whitespace in lstlisting
\usepackage{lstautogobble}

% PDF-compatible landscape mode.
% Makes PDF viewers show the page rotated by 90°.
\usepackage{pdflscape}

% Advanced tables
\usepackage{array}
\usepackage{tabularx}
\usepackage{longtable}

% Fancy tablerules
\usepackage{booktabs}

% Graphics
\usepackage{graphicx}

% Current time
\usepackage[useregional=numeric]{datetime2}

% Float barriers.
% Automatically add a FloatBarrier to each \section
\usepackage[section]{placeins}

% Custom header and footer
\usepackage{fancyhdr}

\usepackage{geometry}
\usepackage{layout}

% Math tools
\usepackage{mathtools}
% Math symbols
\usepackage{amsmath,amsfonts,amssymb}
\usepackage{amsthm}

\DeclarePairedDelimiter{\ceil}{\lceil}{\rceil}
\DeclarePairedDelimiter{\floor}{\lfloor}{\rfloor}

% General symbols
\usepackage{stmaryrd}

\DeclarePairedDelimiter\abs{\lvert}{\rvert}

% Indistinguishable operator (three stacked tildes)
\newcommand*{\diffeo}{% 
  \mathrel{\vcenter{\offinterlineskip
  \hbox{$\sim$}\vskip-.35ex\hbox{$\sim$}\vskip-.35ex\hbox{$\sim$}}}}

% Bullet point
\newcommand{\tabitem}{~~\llap{\textbullet}~~}

\pagestyle{plain}
% \fancyhf{}
% \lhead{}
% \lfoot{}
% \rfoot{}
% 
% Source code & highlighting
\usepackage{listings}

% Coloured boxes!
\usepackage[most]{tcolorbox}
\newtcolorbox{library}[2][]{
	enhanced,
	sharp corners,
	coltitle=black, % title colour
	colbacktitle=black!10!white, % title bg colour
	halign title=center, % title align
	toptitle=1mm, % Top/bottom additional spacing for title
	bottomtitle=1mm,
	fonttitle=\ttfamily,
	colback=white, % body bg colour
	fontupper=\ttfamily,
	title=#2,#1
}

\newtcolorbox{boxcomment}[2][]{
	enhanced,
	colframe=white, % frame colour
	colbacktitle=white, % title bg colour
	halign=center, % body align
	colback=white, % body bg colour
	fonttitle=\ttfamily,
	fontupper=\ttfamily,
	title=#2,#1
}

% SI units
\usepackage[binary-units=true]{siunitx}
\DeclareSIUnit\cycles{cycles}

% Convenience commands
\newcommand{\mailsubject}{41100 - Distributed algorithms - Series 7}
\newcommand{\maillink}[1]{\href{mailto:#1?subject=\mailsubject}
                               {#1}}

% Should use this command wherever the print date is mentioned.
\newcommand{\printdate}{\today}

\subject{41102 - Distributed algorithms}
\title{Series 7}

\author{Michael Senn \maillink{michael.senn@students.unibe.ch} - 16-126-880}

\date{\printdate}

% Needs to be the last command in the preamble, for one reason or
% another. 
\usepackage{hyperref}


\begin{document}
\maketitle


\setcounter{chapter}{6}

\chapter{Series 7}

\section{Worst-case latency of eager reliable broadcast}

In a worst-case execution, all intermediary processes (those broadcasting a
message after they delivered it) will manage to reach exactly one process which
has not delivered the message yet, before crashing. The first (of which all
messages except one are delayed until the end of the execution) and last
process keep running.

In other words:
\begin{itemize}
	\item Let $\Pi = \{p_1, \cdots, p_n\}$
	\item $p_1$ broadcasts $[DATA, p_1, m]$ to all processes, all messages
		except the one to $p_2$ will be delayed until the end of the
		execution.
	\item $p_2$ delivers $[DATA, p_1, m]$ from $p_1$ and broadcasts it to
		all processes
	\item $p_3$ delivers $[DATA, p_1, m]$ from $p_2$ and broadcasts it to
		all processes
	\item $p_2$ crashes
	\item $\cdots$
	\item $p_n$ delivers $[DATA, p_1, m]$ from $p_{n - 1}$ and broadcasts
		it to all processes
	\item $p_{n - 1}$ crashes
\end{itemize}

With $\abs{\Pi} = n$ this will have produced to $O(n^2)$ messages of which
$O(n)$, were delivered, with a total number of $n$ `hops' from original source
to final destination.

\section{Uniform reliable broadcast in the fail-stop model}

\subsection{No strong completness}

If the failure detector does not offer strong completness, then a process may
fail without the failure detector recognizing it.

Assume $p$ is broadcasting a message $m$, and $q$ has failed without $p$'s
failure detector recognizing it. In such a case, $p$ will wait forever for $q$
acknowledge the delivery of $m$, which will never happen. This violate's the
validy property of $p$, as it will never terminate. Since there are extensions
of this execution - namely ones where $q$ is detected as having crashed - where
the property would hold, this is a liveness violation.

\subsection{No strong accuracy}

If the failure detector does not offer strong accuracy, then it might suspect a
correct process of having crashed.

Assume $p$ is broadcasting a message $m$, and $q$ is erroneously detected as
having crashed by all other processes including $p$. Assume further that no
messages from other processes are currently arriving at $q$.

Once all processes have received each others' acnkowledgement that they
$beb$-delivered $m$, they will all $urb$-deliver $m$. If all these processes
then crash, there are processes which $urb$-delivered $m$, yet the correct
process $q$ will never $urb$-deliver $m$, thereby violating the uniform
agreement property. As any extension of an execution where all processes except
$q$ crashed will also have the property violated (since there is none to send
the message to $q$), this is a safety violation.

\section{FIFO broadcast from FIFO links}

A FIFO reliable broadcast has to fulfill properties RB1 through RB4, as well as
FRB5.

As a FIFO perfect link has all the properties of a regular perfect link,
algorithm 3.3 using FIFO perfect links in place of perfect links is guaranteed
to have properties RB1 through RB4.

Consider now process $p$ broadcasting messages $m_1$ and $m_2$ in sequence. If
$m_2$ is subsequently delivered - via any FIFO perfect link - at process $q$,
then the `FIFO delivery' property of a FIFO perfect link guarantees that
message $m_1$ was delivered - via that same link - at process $q$ earlier. From
this property FRB5 follows, as the modified algorithm 3.3 is guaranteed to not
`see' $m_2$ before $m_1$ - at least assuming processes are single-threaded and
event handlers are atomic.

\end{document}
